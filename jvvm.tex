\documentclass[a4paper]{article}

%% Language and font encodings
\usepackage[brazil]{babel}
\usepackage[utf8x]{inputenc}
\usepackage[T1]{fontenc}

%% Sets page size and margins
\usepackage[a4paper,top=3cm,bottom=2cm,left=3cm,right=3cm,marginparwidth=1.75cm]{geometry}

%% Useful packages
\usepackage{amsmath}
\usepackage{graphicx}
\usepackage[colorinlistoftodos]{todonotes}
\usepackage[colorlinks=true, allcolors=blue]{hyperref}
\usepackage{longtable}

\title{IF673 - Lógica para Computação}
\author{João Vítor Valadares de Moraes}

\begin{document}
\maketitle

\section{Introdução}
\textbf{Lógica para Computação (IF - 673)} é uma das cadeiras obrigatórias componentes do curso de Ciência da Computação da Universidade Federal de Pernambuco (UFPE). Correspondente ao 2° período do curso, está inserida na área dos Fundamentos Matemáticos da Computação. A disciplina trabalha o raciocínio dedutivo do aluno no âmbito da Lógica Matemática, estudando as noções de validade e consistência de argumentos utilizando ferramentas da Matemática. 

Resumidamente, como está dito no site da disciplina\cite{SiteDisciplina}, a disciplina aborda as potencialidades e limites do método formal-dedutivo de representação e raciocínio sobre uma "realidade", a fundamentação das noções de prova e refutação da validade de argumentos, e por último, os fundamentos da representação simbólica, e da noção de consequência lógica. 
Para algumas informações extras, como site de monitoria, sistema de avaliação, está disponível o CInWiki da cadeira.\cite{WikiLogica}


\section{Relevância}
O estudo da lógica é de extrema importância para o profissional da área da Computação, pois é a partir dela que ele poderá desenvolver o raciocínio lógico dedutivo, tendo como consequência o desdobramento dos conceitos fundamentais da Ciência da Computação e com isso, se tornar um profissional mais capacitado na área.

Como exemplo, temos o raciocínio dedutivo num sistema formal, elaborado por Gottlob Frege, considerado o mentor da lógica moderna, que está ligado diretamente ao fundamentos lógicos de uma "máquina abstrata de computação efetiva", como por exemplo, a Máquina de Turing, concebida pelo matemático Alan Turing.
Além disso, também é estudado, de forma abstrata e geral, o conceito de máquina de processamento simbólico e as noções de representação e manipulação simbólica, conhecimento que é imprescindível à quem se encontra na área da computação.

A Lógica para Computação, então, acaba se inserindo numa área essencial para o cientista da computação, pois aquele que tiver domínio sobre sua teoria e suas ferramentas estará bem capacitado para seguir profissionalmente na área, já que a partir delas terá o conhecimento necessário para resolução de diversos tipos de problemas computacionais.

\subsection{Pontos positivos}
\begin{enumerate}
    \item Desenvolve o raciocínio lógico do aluno, essencial no decorrer do curso e na carreira profissional;
    \item Aprendizado de métodos para resolução de SAT sem uso de "força bruta";
    \item Proporciona uma base sólida para o estuda de IA;
    \item Auxilia no desenvolvimento de algoritmos.
\end{enumerate}

\subsection{Pontos negativos}
\begin{enumerate}
    \item É de teor completamente teórico.
\end{enumerate}

\clearpage

\section{Relação com outras disciplinas}
A tabela a seguir mostra uma relação entre a cadeira de Lógica para Computação e algumas cadeiras do curso.

\begin{table}[h]
\begin{tabular}{|l|l|}
\hline
\textbf{Disciplina}                                                                    & \textbf{Relação com Lógica para Computação}                                                                                                                                                                                                                                                   \\ \hline
\begin{tabular}[c]{@{}l@{}}IF670 - Matemática Discreta para \\ Computação\end{tabular} & \begin{tabular}[c]{@{}l@{}}Serve de base para a cadeira de lógica,\\ sendo assim um pré-requisito. É responsável\\  pelo estudo das grandezas matemáticas \\ discretas e finitas, como por exemplo \\ números inteiros, grafos, árvores, indução e \\ recursividade.\end{tabular}             \\ \hline
IF689 - Informática Teórica                                                            & \begin{tabular}[c]{@{}l@{}}Como a informática teórica estuda problemas\\ computacionais e classes de  linguagens que \\ podem ser reconhecidas por modelos \\ computacionais simbólicos, acaba sendo \\ necessário o conhecimento da lógica em \\ linguagens simbólicas.\end{tabular}         \\ \hline
\begin{tabular}[c]{@{}l@{}}IF682 - Engenharia de Software e \\ Sistemas\end{tabular}   & \begin{tabular}[c]{@{}l@{}}Tem a disciplina de lógica como \\ pré-requisito, pois é necessário domínio \\ dos conceitos de lógica estudados para a \\ construção de softwares e sistemas, \\ como também de linguagens de programação,\\ que possui relação direta com a lógica.\end{tabular} \\ \hline
\end{tabular}
\end{table}

\section{Referências - Bibliografia Oficial}
Como bibliografia oficial, a cadeira utiliza os seguintes livros:
\begin{enumerate}
    \item Language, Proof and Logic.\cite{barwise1999language}
    \item A Shorter Model Theory.\cite{hodges1997shorter}
    \item Logic and Structure.\cite{van2012logic}
\end{enumerate}

\bibliographystyle{alpha}
\bibliography{jvvm.bib}
\end{document}