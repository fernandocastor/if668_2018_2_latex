\documentclass[10pt]{extarticle}
\usepackage[utf8]{inputenc}
\usepackage[table,xcdraw]{xcolor}
\usepackage[brazil]{babel}
\usepackage{hyperref}
% \usepackage{booktabs}
 \title{ET586 - Estatística e Probabilidade para Computação}
\author{Lucas Emanuel de Farias Mendes}
%s\date{October 2018}
 \usepackage{natbib}
\usepackage{graphicx}
 \begin{document}
 \maketitle
 \section{Introdução} Estatística e Probabilidade para computação é uma disciplina do Centro de Informática (CIn), ministrada atualmente, para o curso de ciência da computação no segundo período, por Renata Souza, para Engenharia da computação no quarto período, por Tsang Ing Ren, e para Sistemas de Informação no segundo período, por Pedro Magalhães. O objetivo da disciplina é dar ao aluno um conhecimento básico da abordagem de diversos aspectos dos campos da probabilidade e da estatística, como o estudo do espaço amostral, das definições de probabilidades, de métodos demonstrativos, entre outros.
 \cite {3}
 \cite {4}
 \cite {5}
 \hspace{\parindent}
 \begin{figure}[h!]
\centering
\includegraphics[width=0.7\textwidth]{Corvette_Sales_1953_to_2013.png}
\caption{exemplo de gráfico usado para representar e facilitar o entendimento de dados estatísticos, tem-se nesta figura a representação gráfica dos dados de vendas de certos modelos de carros em um determinado período. \newline}

\url{https://bit.ly/2qdcrMw |}
\href{https://commons.wikimedia.org/wiki/File:Corvette_Sales_1953_to_2013.png}{Autor: Mark Lawrence\newline Licensa: Creative Commons Attribution-Share Alike 3.0 Unported}
\label{fig:Corvette_Sales_1953_to_2013}\newline\newline
\end{figure}
 \begin{figure}[h!]
\centering
\includegraphics[width=0.7\textwidth]{Event_Tree_Diagram.JPG}
\caption{Tipo de arvore de eventos, estrutura utilizada para demonstrar probabilidades simples, da ocorrência de certos eventos sucessivos. \newline}

\url{https://bit.ly/2EIcMAI |}
\href{https://commons.wikimedia.org/wiki/File:Event_Tree_Diagram.JPG}{Autor: Desconhecido\newline Licensa: Creative Commons Attribution-Share Alike 3.0 Unported}
\label{fig:Corvette_Sales_1953_to_2013}
\end{figure}

 \section{Relevância}
 \hspace{\parindent}A cadeira é extremamente importante para os três cursos, por introduzir conceitos chave que são de extrema relevância para um aluno que visa seguir na profissão, pois no mercado de trabalho, o conhecimento do estudo das probabilidades e da compreensão correta de dados estatísticos, é um grande pré-requisito de diversas empresas, saber o impacto no público alvo que pode ser causado por certas alterações na usabilidade de um programa, entre vários exemplos, além desse aspecto, é extremamente válida também a analise da importância para as futuras disciplinas que os alunos ainda vão ter, pois o conhecimento adquirido é aplicado em diversas cadeiras posteriores.  Após conversar com um aluno da disciplina do período 2018.2, e analisar o cronograma disponibilizado no site oficial, pôde ser visto que a cadeira tem uma abordagem um pouco mais teórica do que seria agradável para os alunos de computação, por serem mostradas poucas aplicações práticas do que se estuda, porém os pontos positivos são sem dúvidas, maiores. 
 \cite {1}
 \cite {2}
 \cite {3}
 \cite {4}
 \newpage
 
 \section{Relação com outras disciplinas}
 
\begin{table}[h!]
\caption{\newline}
\label{my-label}
\begin{tabular}{ll}
\multicolumn{1}{c}{\textbf{CADEIRA}}                                                                                                                  & \multicolumn{1}{c}{\textbf{RELAÇÃO}}                                                                                                                                                                                                         \\ \hline
\multicolumn{1}{|l}{\cellcolor[HTML]{EFEFEF}{\color[HTML]{000000} \begin{tabular}[c]{@{}l@{}}MA026: CALCULO  INTEGRAL \\ E DIFERENCIAL\end{tabular}}} & \multicolumn{1}{l|}{\begin{tabular}[c]{@{}l@{}}Essa cadeira do primeiro período\\ é um pré requisito da disciplina, \\ muito relacionada por introduzir \\ ao aluno, conhecimentos necessários\\ para o estudo da estatística.\end{tabular}} \\ \hline
\multicolumn{1}{|l}{\cellcolor[HTML]{EFEFEF}{\color[HTML]{000000} IF702: REDES NEURAIS}}                                                              & \multicolumn{1}{l|}{\begin{tabular}[c]{@{}l@{}}A relação existe pela constante \\ necessidade,nesta disciplina, da\\ capacidade de interpretar dados\\  e problemas.\end{tabular}}                                                           \\ \hline
\multicolumn{1}{|l}{\cellcolor[HTML]{EFEFEF}{\color[HTML]{000000} IF689: INFORMÁTICA TEÓRICA}}                                                        & \multicolumn{1}{l|}{\begin{tabular}[c]{@{}l@{}}Relacionada pelos tipos de problemas \\ introduzidos e como interpreta-los \\ com base matemática.\end{tabular}}                                                                              \\ \hline
\multicolumn{1}{|l}{\cellcolor[HTML]{EFEFEF}{\color[HTML]{000000} \begin{tabular}[c]{@{}l@{}}IF699: APRENDIZAGEM DE \\ MÁQUINA\end{tabular}}}         & \multicolumn{1}{l|}{\begin{tabular}[c]{@{}l@{}}Ambas estudam análise de problemas \\ com o objetivo de achar um padrão.\end{tabular}}                                                                                                      \\ \hline
\multicolumn{1}{|l}{\cellcolor[HTML]{EFEFEF}{\color[HTML]{000000} IF797: OTIMIZAÇÃO}}                                                                 & \multicolumn{1}{l|}{\begin{tabular}[c]{@{}l@{}}Existe a relação pois, constantemente,\\ ambas analisam gráficos diversos, \\ com diferentes objetivos, mas com os\\ mesmos princípios.\end{tabular}}                                         \\ \hline
\end{tabular}
\end{table}
 \cite{4}
\cite{5}
\cite{3}
\bibliographystyle{plain}
\bibliography{references}
 \end{document}