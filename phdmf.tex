\documentclass{article}
\usepackage[utf8]{inputenc}
\usepackage{cite}
\usepackage{url}
\title{IF685 - Gerenciamento de Dados e Informação}
\author{Pedro Henrique Dias Monte de Freitas}
\date{Outubro de 2018}

\usepackage{natbib}
\usepackage{graphicx}

\begin{document}

\maketitle

\section{Introdução}
Gerenciamento de Dados e Informação \cite{site}, disciplina ofertada no 4º período do curso de Ciência da Computação tem como objetivo oferecer aos alunos conhecimento geral sobre banco de dados, ou seja, na disciplina é visto desde modelagem de dados até linguagem de 4ª geração - tipo "especial" de linguagem que combina estruturas de controle de linguagem de programação permintindo a manipulação de elementos em um Banco de Dados. A disciplina está dividida em alguns tópicos, sendo eles:

\begin{enumerate}
    \item Modelagem de Dados \cite{navathe}
    \item Banco de Dados Relacional
    \item SQL \citep{navathe}
    \item Linguagem de 4ª Geração - PL
    \item Sistema Objeto-Relacionais - OR
\end{enumerate}

\section{Relevância}
Dado o momento em que vivemos, de tudo - ou quase - é possível extrair dados que após analisados pode-se deduzir informação útil. A informação extraída do dado pode gerar "conhecimento" que é em suma uma informação adicional a partir desse pode-se então:
\begin{enumerate}
    \item transformar dados em conhecimento
    \item extrair informação de informação que já existem
    \item adquirir novos conhecimentos
\end{enumerate}

Todo esse processo ajuda na tomada de decisão de uma empresa, por exemplo. Porém, não "apenas" na tomada de decisão. Como dito antes, de quase tudo é possível extrair alguma informação, dito isso, há necessidade de fazer sistemas seguros\cite{laudon}. Sendo assim, direta ou indiretamente a disciplina auxilia nesse processo também. 

\section{Relação com outras disciplinas}

\begin{table}[h]
 \centering
 {\renewcommand\arraystretch{1.25}
 \begin{tabular}{ l l }
  \cline{1-1}\cline{2-2}  
    \multicolumn{2}{|p{12cm}|}{Gerenciamento de Dados x Outras Disciplinas \centering }
  \\ 
  \cline{1-1}\cline{2-2}  
    \multicolumn{1}{|p{5cm}|}{Disciplina \centering } &
    \multicolumn{1}{p{7cm}|}{Relação \centering }
  \\ 
  \cline{1-1}\cline{2-2}  
    \multicolumn{1}{|p{5cm}|}{IF672 - Algoritmos e Estrutura de Dados \centering } &
    \multicolumn{1}{p{7cm}|}{Com o objetivo de fazer com que os alunos aprendam a escrever programas mais eficientes essa disciplina tem ligação com GDI}
  \\ 
  \cline{1-1}\cline{2-2}  
    \multicolumn{1}{|p{5cm}|}{IF692 - Projeto de Banco de Dados \centering } &
    \multicolumn{1}{p{7cm}|}{Em suma, a disciplina põe em prática todos os conceitos aprendidos em GDI.}
  \\ 
  \cline{1-1}\cline{2-2}  
    \multicolumn{1}{|p{5cm}|}{IF693 -  Sistema de Gerenciamento de Banco de Dados \centering } &
    \multicolumn{1}{p{7cm}|}{É possível dizer que, GDI tem um "foco" mais geral da área de "Dados", mas a disciplina de SGBD, pode-se dizer está baseado no aprendizado "aprofundado" em SGBD's, ou seja, desde uma visão mais geral até técnicas mais específicas.}
  \\ 
  \cline{1-1}\cline{2-2}  
    \multicolumn{1}{|p{5cm}|}{IF694 - Banco de Dados Distribuídos e Móveis \centering } &
    \multicolumn{1}{p{7cm}|}{Assim como na disciplina anterior, GDI tem relação com BDDM pois GDI é fundamental visto que, tem-se uma visão mais ampla da área de Banco de Dados. Pode-se dizer que BDDM é uma "àrea" específica da área de BD. }
  \\
  \cline{1-1}\cline{2-2}  
    \multicolumn{1}{|p{5cm}|}{IF695 - Banco de Dados Avançados \centering } &
    \multicolumn{1}{p{7cm}|}{Nessa disciplina será estudado: 
- BD Multidimensionais e OLAP
- BD Espaciais  			
- BD NoSQL  			
Ou seja, como nas duas disciplinas anteriores, GDI é fundamental para melhor aprendizado dessa disciplina.}
  \\ 
  \hline
\end{tabular}}
\end{table}

Além dessas matérias que compõe a tabela, há algumas outras que tem alguma ligação com GDI principalmente pelo fato de GDI dar uma base sólida na área de Gerenciamento de Dados. Segue as matérias: IF696 - Integração de Dados e Warehousing, IF698 - Seminário em Gerenciamento de Dados e Informação, IF962 - Recuperação de Informação, IF714 - Sistema de Informação e IF799 - Prog. Declarativa BD Inteligentes.

\bibliographystyle{plain}
\bibliography{references}

\end{document}
