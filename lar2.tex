\documentclass[a4paper,10pt]{extarticle}
\usepackage[brazilian]{babel}
\usepackage[utf8]{inputenc}

\title{IF689 - Informática Teórica}
\author{Luan Alves Rodrigues }
\date{22 de Outubro de 2018}

\usepackage{natbib}
\usepackage{graphicx}

\begin{document}

\maketitle

\section{Introdução}
A cadeira de Informática Teórica do curso de Ciência da Computação do Cin, ministrada pelo professor Fred Freitas, abrange a área de teoria da computação. Nessa disciplina os alunos irão aprender sobre assuntos como Autômatos Finitos(que é um modelo computacional com pouca memória disponível capaz de realizar ações como abrir uma porta automática), Autômatos com pilha(que são parecidos com os Autômatos Finitos, mas contam com uma pilha que provê memória adicional) e Máquinas de Turing. O principal livro dessa cadeira é o "Introdução à Teoria da Computação" \citep{MichaelSipser}.

\begin{figure}[h!]
\centering
\includegraphics[scale=0.3]{1}
\caption{Alan Turing\citep{Imagem}}
\label{fig:1}
\end{figure}


\section{Relevância}
Essa disciplina é importante pois ela fornece aos alunos importantes ferramentas conceituais que os alunos irão usar durante o curso, como o uso de autômatos finitos quando se está lidando com a busca de cadeias e casamento de padrões. A cadeira também é necessária pois ela aumenta sua habilidade de pensar pra resolver um problema e saber quando ele não foi resolvido de forma correta, além de aumentar seu senso estético para poder construir sistemas mais bonitos.

\section{Relação com outras disciplinas}
\begin{table}[h]
 \centering
 {\renewcommand\arraystretch{1.25}
 \caption{A sample table}
 \begin{tabular}{ l l }
  \cline{1-1}\cline{2-2}  
    \multicolumn{1}{|p{3.850cm}|}{Disciplina \centering } &
    \multicolumn{1}{p{4.217cm}|}{Relação \centering }
  \\  
  \cline{1-1}\cline{2-2}  
    \multicolumn{1}{|p{3.850cm}|}{IF778- Seminário em Informática Teórica} &
    \multicolumn{1}{p{4.217cm}|}{Na disciplina os alunos irão participar e apresentar seminários sobre informática teórica }
  \\  
  \cline{1-1}\cline{2-2}  
    \multicolumn{1}{|p{3.850cm}|}{IF769- Teoria da Recursão} &
    \multicolumn{1}{p{4.217cm}|}{  Nessa disciplina os alunos também irão estudar sobre Máquina de Turing}
  \\ 
  \cline{1-1}\cline{2-2}  
    \multicolumn{1}{|p{3.850cm}|}{IF776-Tópicos Avançados Inf. Teórica} &
    \multicolumn{1}{p{4.217cm}|}{ Aqui os alunos irão ver técnicas mais avançadas da informática teórica}
  \\ 
  \hline

 \end{tabular} }
\end{table}


\bibliographystyle{plain}
\bibliography{references}
\end{document}
