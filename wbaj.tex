\documentclass[10pt]{article}

%% Language and font encodings
\usepackage[brazil]{babel}
\usepackage[utf8]{inputenc}
\usepackage{graphicx}
\usepackage{wrapfig}
\usepackage[export]{adjustbox}
\usepackage{subcaption}
%% Sets page size and margins
\usepackage[a4paper,top=2cm,bottom=2cm,left=3cm,right=3cm,marginparwidth=1.75cm]{geometry}


\title{IF768 - Teoria dos Grafos}
\author{Wilton de Barros Araújo Júnior}
\date{October, 2018}

\begin{document}
\maketitle

\section{Introdução}
A disciplina \textbf{IF768 Teoria dos Grafos} participa da grade curricular do curso de Ciência da Computação como eletiva e possui carga horária semestral de 75h. Atualmente não possui oficialmente outra disciplina como pré-requisito nem como có-requisito, embora seja necessário ter uma boa base em matemática discreta, álgebra, lógica e algoritmo para cursá-la de forma a extrair o máximo de seu conteúdo. Sua grande contribuição para informática é fornecer subsídios para resolução de problemas combinatoriais e ela enquadra-se na seguinte área do conhecimento:\\
\\
\begin{tabular}{l l l|l}
     & GRANDE ÁREA: Ciências Exatas e da Terra.\\
     & ÁREA: Ciência da Computação.\\
     & SUBÁREA: Matemática da Computação.\\
     & ESPECIALIDADE: Otimização Combinatória.\\
     & \footnotesize{Classificação adotada pelo Conselho Nacional de Desenvolvimento Científico e Tecnológico (CNPq).}\\
\end{tabular}
\\
\\

\section{Um pouco de História...}
Na cidade de Königsberg, antigo território da Prússia, atual Kaliningrado (Rússia), existem 2 ilhas no rio Pregel conectadas entre si e com as margens por 7 pontes. Na época discutia-se sobre a possibilidade de atravessar todas as pontes sem repetir nenhuma. Apenas em 1735, o físico e matemático suiço Leonard Euler provou não ser possível tal feito usando um raciocínio muito simples: transformou os caminhos em linhas e as intersecções em pontos, criando possivelmente o primeiro grafo da história.
\\

\begin{figure}[h]
 
\begin{subfigure}{0.5\textwidth}
\includegraphics[width=0.9\linewidth, height=5cm]{Solutio.png} 
\caption{Primeira página.}
\label{fig:subim1}
\end{subfigure}
\begin{subfigure}{0.5\textwidth}
\includegraphics[width=0.9\linewidth, height=5cm]{Pontes.png}
\caption{Pontes sobre o rio Pregel.}
\label{fig:subim2}
\end{subfigure}
 
\caption{Trabalho apresentado por Leonard Euler na \textbf{the St. Petersburg Academy} em 1735 e tido como o seu mais famoso artigo. É considerada a mais antiga referência publicada em topologia e teoria dos grafos. }
\label{fig:image2}
\end{figure}

\newpage
\section{Relevância e Dificuldades}
Se pudéssemos resumir a uma só palavra a funcionalidade dos grafos provavelmente ela seria OTIMIZAÇÃO. Ferramenta essencial da otimização combinatória, os grafos contribuem para simplificação de modelos de sistemas em projetos de dimensionamento e alocação de redes, estrutura de dados e algoritmos. Problemas de fluxo máximo, caminho mínimo e emparelhamento máximo podem ser abordados com eficiência usando essa teoria.\\

Uma dificuldade encontrada pelos alunos para o acompanhamento optimum da disciplina está na base matemática que o aluno traz consigo ao matricular-se. É desejável que se tenha razoável conhecimento das disciplinas básicas citadas na introdução deste trabalho que o permitam manipular ferramentas como prova por indução, prova por contradição, lógica computacional, dentre outros.

\section{Ementa da disciplina \cite{2} \cite{1} \cite{disciplina}}
Grafos;\\
Subgrafos e grafos orientados;\\
Florestas e árvores;\\
Busca em Grafos, conectividade e cortes;\\
Árvore geradora, distâncias, fluxo em rede e emparelhamento;\\
Problemas intratáveis.\\

\section{Relação com outras disciplinas:}

\begin{tabular}{ |p{7cm}||p{7cm}|}
\hline
|\large{\textsc{Disciplina}}|& |\large{\textsc{Relação}}|\\[0.1ex]
\hline

IF669 - Introdução à computação & Base matemática para acompanhamento da disciplina.\\
\hline
IF673 - Lógica para Computação & Base lógica para acompanhamento da disciplina.\\
\hline 
IF767 - Processamento Cadeia Caracteres & Grafo dá suporte.\\
\hline
IF767 - Processamento Cadeia Caracteres & Grafo dá suporte.\\
\hline
\end{tabular}\\

\bibliographystyle{plain}
\bibliography{references}

\end{document}