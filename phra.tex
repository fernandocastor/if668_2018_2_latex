\documentclass{article}
\usepackage[utf8]{inputenc}
\usepackage[brazil]{babel}
\usepackage[T1]{fontenc}
\usepackage{array}
\usepackage{varwidth}


\title{IF685 - Gerenciamento de Dados e Informação}
\author{Pedro Henrique Ralph arruda }
\date{October 2018}

\usepackage{natbib}
\usepackage{graphicx}

\begin{document}

\maketitle

\section{Introdução}
Gerenciamento  de Dados e Informação serve como uma introdução aos conceitos e principais áreas de Sistemas de Informação.
O grande foco da disciplina é,como se pode imaginar pelo nome da cadeira, o gerenciamento e manutenção de bancos de dados conectados à uma rede. 
Além disso a disciplina entra no campo de linguagens de programação feitas para objetivo de criar sistemas de dados cada vez mais optimizados. 

No CIn a cadeira cobre os tópicos de:
\begin{itemize}
\item Modelagem de Dados : Onde os alunos aprendem a transformar informações do mundo real em dados que podem ser analisados e armazenados.
\item Bancos de Dados : Com o foco na teoria algébrica e e técnicas de esquematização esta parte também prepara os alunos para o uso de SQL.
\item Dados Semi-Estruturados : Esta parte foca no uso de XML para trabalhar com dados irregulares e voláteis.
\item Aplicações desses assuntos .
\end{itemize}


\begin{figure}[h!]
\centering
\includegraphics[scale=0.30]{photo.jpeg}

\label{fig: https://www.pexels.com/photo/macbook-pro-on-brown-table-139387/ Licença CC0}
\end{figure}

\section{Relevância}
A disciplina oferece os cruciais primeiros passos e eventualmente conceitos avançados para o ramo de Sistemas de Informação, a sua existência no currículo é fundamental para os alunos que seguem esse caminho e realmente não pode ser substituída ou removida sem causar distúrbios e modificações pela grade curricular inteira.

As técnicas aprendidas nesta cadeira serão usadas pelo resto da vida acadêmica e de trabalho do aluno.Desdos menores sistemas a grandes servidores.


\begin{varwidth}[t]{.5\textwidth}
Pontos positivos  :
\begin{itemize}
\item Ótima introdução e fundamentação teórica
\item Ensina o uso de Oracle e SQL
\end{itemize}
\end{varwidth}
\hspace{4em}
\begin{varwidth}[t]{.5\textwidth}
Pontos negativos :
\begin{itemize}
\item A carga teórica no começo da cadeira pode ser maçante
\item A cadeira pode não atrair alunos que não já tenham interesse pelo assunto
\end{itemize}
\end{varwidth}


\section{Relação com outras Disciplinas}
\begin{table}[!h]
\begin{tabular}{|m{5cm}l|l|}
\hline
IF672 - Algoritmos e Estrutura de Dados & \begin{tabular}[c]{@{}l@{}}Disciplina fundamental para o progresso e\\ bom entendimento da cadeira e isso é evidente\\ por ser um pré-requisito para a matricula\\ na cadeira de Gerenciamento de Dados e Informação.\end{tabular} \\ \hline
ET568 - Estatística Probabilidade p/ Programação & \begin{tabular}[c]{@{}l@{}}Uma cadeira sobre dados não seria completa \\ sem um bom entendimento sobre estatística.\end{tabular} \\ \hline
IF682 - Engenharia de Software e Sistemas & \begin{tabular}[c]{@{}l@{}}A habilidade de criar de sistemas funcionais e estáveis e\\mante-los é importante num ambiente de banco de dados. \end{tabular} \\ \hline
IF681 - Interfaces Usuário-Máquina & \begin{tabular}[c]{@{}l@{}}Qualquer quantidade de informação é inútil sem\\ um jeito de acessa-la e ser capaz de entende-la facilmente.\end{tabular} \\ \hline
\end{tabular}
\end{table}
\nocite{0001}
\nocite{0002}
\nocite{0003}
\nocite{0004}
\bibliographystyle{plain}
\bibliography{references}
\end{document}
