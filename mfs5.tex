\documentclass{article}
\usepackage[utf8]{inputenc}

\title{IF690-História e Futuro da Computação}
\author{Matheus Felipe da Silva}
\date{Outubro 2018}

\usepackage{natbib}
\usepackage{graphicx}
\usepackage[brazil]{babel}
\usepackage[utf8]{inputenc}
\usepackage[T1]{fontenc}

\begin{document}

\maketitle

\section{Introdução}
Nesta cadeira, mais conhecida como "HFC", estuda-se alguns eventos que impulsionaram a história da tecnologia, os processos evolutivos que fizeram da computação o que é hoje e o que se espera da tecnologia nos próximos anos.
O curso é ministrado pelo professor Germano Crispim e as referências oficiais da cadeira são os livros:  História da Computação \cite{historia2}, História da Computação: teoria e tecnologia \cite{historia}, Out of their minds: The Lives and Discoveries of 15 Great Computer Scientists \cite{historia1}. 

\begin{figure}[h!]
\centering
\includegraphics[scale=0.16]{computador.jpg}
\caption{Evolução dos Computadores}
\label{fig:computador}
\end{figure}

\section{Relevância}
Pelo fato de a computação em si ser uma ciência muito recente, é necessário que se estude os aspectos que a regem. Para isso, é muito importante conhecer seu passado e compreender os fatores que a expandem. Por isso, a disciplina de História e Futuro da Computação, está incluída no perfil curricular obrigatório da UFPE, paras os cursos de Ciência da Computação e Engenharia da Computação.
\subsection{Pontos Positivos}
\begin{enumerate}
    \item 
    Ajudar o aluno a compreender a história da área que o mesmo escolheu para seguir carreira.
    \item
    Compreender os erros do passado e entender o quão esses erros foram importantes para o desenvolvimento das áreas de TI.
    
\end{enumerate}
\subsection{Pontos Negativos}
\begin{enumerate}
    \item 
    Ao meu ver, não existe pontos negativos.
\end{enumerate}
\section{Relação com outras cadeiras}
\begin{table}[h!]
\begin{tabular}{|l|l|}
\hline
Disciplina                     & Relação                                                                                                                                                                                 \\ \hline
IF-668 Introdução a Computação & \begin{tabular}[c]{@{}l@{}}As duas cadeiras envolvem aspectos focados principalmente \\ em assuntos históricos, deixando um pouco de lado\\  aspectos práticos e técnicos.\end{tabular} \\ \hline
\end{tabular}
\end{table}

\bibliographystyle{plain}
\bibliography{references.bib}


\end{document}
