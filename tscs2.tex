\documentclass{article}
\usepackage[utf8]{inputenc}
\usepackage[brazil]{babel}
\usepackage{indentfirst}
\usepackage{natbib}
\usepackage{color}

\title{IF754 - Computação Musical e Processamento de Som}
\author{Tiago Campêlo}

\usepackage{graphicx}

\begin{document}

\maketitle

\section{Introdução}
Computação Musical e Processamento de Som é uma cadeira disponibilizada pelo CIn, que estuda os elementos da computação relacionados à forma sonora.

Esta cadeira está na área da computação de Processamento de Dados, que engloba o processamento de som e o processamento gráfico, e pretende disponibilizar ao aluno um maior conhecimento sobre as estruturas de áudio, introduzi-lo a técnicas de manipulações digitais em extensões musicais e ensiná-lo algoritmos para a síntese e processamento de som.

Os tópicos abordados na cadeira baseiam-se em fundamentos de som e acústica, no processamento e a síntese de áudio, e na recuperação de informação sonora. Porém, há diversos tópicos avançados que são decididos pelos alunos em conjunto. Esses tópicos estão relacionados ao estudo de multimídia, internet, à ferramentas de programação sonoras, inteligência artificial e aplicações em áreas específicas, como a criação de softwares educativos, e automação de estúdio.

\section{Relevância}
O estudo do processamento de som é muito importante, pois assim como processamento gráfico, essas áreas estão presentes no dia-a-dia de todos. A computação musical ajuda-nos a resolver dos problemas mais simples, como a necessidade de compartilharmos uma música, até os mais complexos, como comunicar-se com alguém que fala outra língua, estando do outro lado do mundo.

A existência da cadeira, portanto, tem o objetivo de disponibilizar ainda mais ferramentas para que o estudante de computação resolva qualquer tipo de problema relacionado a problemas sonoros.

\subsection{Pontos positivos}
\begin{itemize}
    \item Disponibiliza aos alunos interessados na parte acústica um estudo avançado sobre os tópicos relacionados à computação sonora;
    \item Aplica conhecimentos físicos na área da computação, aumentando a possível área de estudo dos que pagam a cadeira;
    \item Dá liberdade aos alunos quanto os tópicos avançados que serão estudados, na cadeira, formando assim um curso único a cada semestre;
    \item Serve de grande incentivo para os interessados a seguir carreira de jogos, pois a parte musical desses é de grande importância no desenvolvimento dos mesmos;
    \item Como diz o próprio professor da cadeira, Geber Ramalho, \citep{SiteDisciplina}, a mesma dá acesso a todos os interessados na área, não apenas a músicos, como alguns esperam por conta do nome.
\end{itemize}

\section{Relação com outras disciplinas}

\begin{table}[h!]
\begin{tabular}{|c|c|}
\hline
Processamento Gráfico - IF680 & \begin{tabular}[c]{@{}c@{}}Assim como Processamento de Som, a cadeira \\ de Processamento Gráfico está inserida na na grande \\ área  da computação de Processamento de Dados, \\ que oferecem um melhor conhecimento da área.\end{tabular} \\ \hline
Algoritmo e Estrutura de Dados - IF672 & \begin{tabular}[c]{@{}c@{}}A cadeira de algoritmos está relacionada à disciplina \\ de processamento de som por focar no desenvolvimento\\ de programas otimizados.\citep{SiteAlgoritmos}{} Essa cadeira acaba sendo um \\ pré-requisito para várias outras, pois ajuda na criação\\ de programas melhores.\end{tabular} \\ \hline
Física para computação - FI582 & \begin{tabular}[c]{@{}c@{}}Física para computação explora áreas da física ondulatória,\\ que estão diretamente relacionadas com a disciplina de \\ Processamento de Som.\end{tabular} \\ \hline
Seminário em Inteligência Artificial - IF707\citep{SiteIA} & \begin{tabular}[c]{@{}c@{}}Na disciplina de Processamento de Som estudam tópicos \\ avançados relacionados a inteligência artificial, que \\ exploram meios de reconhecimento de som, padrões\\ musicais, e sistemas de acompanhamento automático.\end{tabular} \\ \hline
\end{tabular}
\end{table}

\bibliographystyle{plain}
\bibliography{references.bib}
\end{document}
