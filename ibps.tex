\documentclass[10pt, a4paper]{article}
\usepackage[utf8]{inputenc}
\usepackage[brazilian]{babel}

\title{IF686 - Paradigmas de Linguagens Computacionais}
\author{Igor Simoes}
\date{October 2018}

\usepackage{natbib}
\usepackage{float}
\usepackage{tabularx}
\usepackage{graphicx}

\begin{document}

\maketitle

\section{Introdução}
Esta disciplina está principalmente inserida na área de desenvolvimento de software e suas ramificações. Seu objetivo é melhorar a compreensão dos estudantes sobre o padrão de construção das variadas linguagens de programação.\cite{plpcin} Os paradigmas mais comuns aprendidos são: o imperativo, em que o programador instrui a máquina como mudar o estado de sua memória, usando Java como exemplo; o declarativo, em que o programador especifica os resultados desejados, mas não como obtê-los, usando Haskell como exemplo.\cite{plpubuk}\cite{plprt}
\begin{figure}[H]
\centering
\includegraphics[width=8.0cm]{LinguagensdeProgramacao.png}
\caption{Acima temos algumas das várias linguagens de programação.}
\label{fig:linguagensdeprogramacao}
\end{figure}

\section{Relevância}
Para uma empresa ou programa ser bem sucedido entre os outros, é urgente ter algo que o diferencie de seus concorrentes. Dito isto e dados problemas a serem resolvidos, aprender quais línguas são mais vantajosas na resolução é fundamental para que a diferenciação possa ser alcançada.\cite{fcv} Esse é o lado positivo dessa disciplina: ser capaz de dizer se a linguagem "a" é mais útil que a linguagem "b", ou o oposto, dependendo da sua necessidade em um determinado momento. No entanto, a maior desvantagem é que você não é sendo ensinado como programar nessas linguagens, além de Haskell e Java, e isso pode ser um grande problema.

\section{Relação com outras disciplinas}

\begin{table}[H]
\centering
\begin{tabular}{|p{3.2cm}|p{7.8cm}|}
\hline
 Disciplinas & Assuntos Compartilhados \\ \hline
 IF669-Introdução à Programação & Antes de aprender quais línguas são mais vantajosas do que outras, o aluno precisa, antes de mais nada, aprender a programar.\\ \hline
 IF672-Algoritmos e Estrutura de Dados & O aluno aprende a escrever programas mais eficientes, que também se beneficiam da linguagem de programação mais correta para resolver o problema.\\ \hline
 IF708-Programação Funcional & Esta disciplina ensina conceitos avançados não vistos antes e outras linguagens funcionais.\\ \hline
 IF710-Programação com Componentes & Estuda novos conceitos da programação Orientada a Objetos.\\ \hline
 IN045-Tópicos Avançados em Linguagens Computacionais &  Estudo de técnicas avançadas em Linguagens Computacionais.\\
 \hline
\end{tabular}
\caption{Acima temos disciplinas que são requisito para essa disciplina ou que têm essa disciplina como pré-requisito.}
\label{tab:disciplinaeassuntoscompartilhados}
\end{table}

\bibliographystyle{plain}
\bibliography{ibps}
\end{document}
