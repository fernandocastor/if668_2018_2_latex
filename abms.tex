\documentclass[10pt,a4paper]{article}

\usepackage[brazil]{babel}
\usepackage[utf8x]{inputenc}
\usepackage[T1]{fontenc}

\usepackage[a4paper,top=3cm,bottom=2cm,left=3cm,right=3cm,marginparwidth=1.75cm]{geometry}

\usepackage{amsmath}
\usepackage{graphicx}
\usepackage[colorinlistoftodos]{todonotes}
\usepackage[colorlinks=true, allcolors=blue]{hyperref}



\title{IF684 - Sistemas Inteligentes}
\author{Arthur Bernardo Marques da Silva}
\date{21 de Outubro de 2018}

\begin{document}

\maketitle

\begin{figure}[h]
\centering
\includegraphics[width=0.5\textwidth]{ai.jpg}
\caption{\label{fig:ai}CC0 Creative Commons}
\end{figure}

% link da imagem :https://pixabay.com/pt/intelig%C3%AAncia-artificial-ai-rob%C3%B4-2228610/
% licença CC0 Creative Commons

\section{Introdução}
\paragraph{}O objetivo da disciplina remete ao estudo de técnicas computacionais que apresentem características de aprendizagem automática, fornecer uma visão geral da área de aprendizagem de máquina, estudar métodos e técnicas de aprendizagem de \textit{simbólica, conexionista, evolucionista} e estudar aspectos teóricos e práticos. Os livros base da disciplina são \cite{AI:ModernApproach} e \cite{AI:NewSynthesis}, entre outros livros e artigos para complementar ainda mais o aprendizado que podem ser encontrados no site principal da cadeira \cite{SiteCadeira}. \paragraph{} Sistemas Inteligentes insere-se na área da Inteligência Artificial que segundo Elaine Rich é uma área de pesquisa que investiga formas de habilitar o computador a realizar tarefas nas quais, até o momento, o ser humano tem um melhor desempenho. 

\section{Relevância}
\paragraph{}Esta disciplina está presente no curso de Ciência da Computação para fornecer ao aluno uma visão geral de alguns métodos e técnicas bastante difundidos dentro da Inteligência Artificial de maneira que ele seja capaz de modelar problemas que necessitam de soluções com algoritmos de grande complexidade e identificando as técnicas mais apropriadas para sua solução. Alguns pontos positivos no que diz respeito ao estudo desta disciplina são:

\begin{enumerate}
\item O aluno é inserido em um dos ramos da computação que mais crescem cientificamente e que está em destaque na atualidade. Dando uma visão geral do que venha ser Sistemas Inteligentes e consequentemente inserindo-o no mundo da Inteligência Artificial.
\item Com o estudo desta disciplina o aluno irá aprender técnicas computacionais avançadas para solucionar melhoras e/ou problemas em diversos segmentos da tecnologia com algoritmos eficientes de busca e otimização de alta complexidade.
\item Atualmente, são várias as aplicações na vida real da Inteligência Artificial: Jogos, programas de computador, aplicativos de segurança para sistemas informacionais, robótica, dispositivos para reconhecimento de escrita da mão e de voz, programas de diagnósticos médicos e muito mais, fazendo com que o aluno tenha um grande leque de oportunidades.
\end{enumerate}

\section{Relação com outras disciplinas}

\begin{table}[h]
\begin{tabular}{|p{7cm}|p{7cm}|}
\hline

IF672 - Algoritmos e Estruturas de Dados  & \begin{tabular}[c]{@{}p{7.121cm}@{}}Estudamos estruturas de dados para que possamos aprender a escrever programas mais eficientes. E isso é peça chave quando trata-se de Inteligência Artificial.\end{tabular} \\ 
\hline
IF673 - Lógica para Computação  & \begin{tabular}[c]{@{}p{7.121cm}@{}}O aprendizado da lógica auxilia os estudantes no raciocínio, na compreensão de conceitos básicos, na verificação formal de programas e melhor os prepara para o entendimento do conteúdo de tópicos mais  avançados.\end{tabular} \\ 
\hline
IF793 - Projeto Implementação de Jogos 2D & \begin{tabular}[c]{@{}p{7.121cm}@{}}Quando se trata de Games, IA se faz presente. Para poder cursar a referida disciplina, o aluno deverá dominar os fundamentos de programação e inteligência artificial.\end{tabular} \\ 
\hline
IF962 - Recuperação de Informação & \begin{tabular}[c]{@{}p{7.121cm}@{}}A Inteligência Artificial é usada na recuperação de informação. Sistemas Inteligentes é uma cadeira necessária para o estudo dessa matéria.\end{tabular} \\ 
\hline
IF699 - Aprendizagem de Máquina & \begin{tabular}[c]{@{}p{7.121cm}@{}}A disciplina tem como objetivo estudar métodos e algoritmos que obtém conhecimento a partir da análise de bases de dados. Conceitos de IA são usados no estudo dessa matéria.\end{tabular} \\ 
\hline
IF706 - Tópicos Avanc. Inteleg. Artificial & \begin{tabular}[c]{@{}p{7.121cm}@{}}Nesta cadeira o aluno estudará técnicas avançadas em inteligência artificial, permitindo-o conhecer o estado da arte nesta área de pesquisa.\end{tabular}\\ 
\hline
IF707  - Semin. em Inteligência Artificial & \begin{tabular}[c]{@{}p{7.121cm}@{}}Nesta cadeira são abordados temas específicos sobre Sistemas Inteligentes abordando conceitos essenciais sobre a Inteligência Artificial, fazendo com que o aluno tenha uma visão bastante ampla do conceito de IA e ao mesmo tempo o incentivo à pesquisa e contato com alunos de pós-graduação.\end{tabular} \\ 
\hline
\end{tabular}
\end{table}

\bibliography{main}
\bibliographystyle{plain}
 

\end{document}
