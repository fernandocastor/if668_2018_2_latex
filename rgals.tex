\documentclass[10pt]{article}
\usepackage[utf8]{inputenc}
\usepackage[brazil]{babel}
\usepackage{float}

\title{Introdução à Multimídia}

\author{rgals }
\date{Outubro 2018}

\usepackage{natbib}
\usepackage{graphicx}

\begin{document}

\maketitle
\begin{figure}[h!]
\centering
\includegraphics[scale=1.2]{VR.jpg}

\label{fig:universe}
\end{figure}

\section{Introdução}
A disciplina de Introdução à Multimídia apresenta os conceitos básicos e as ferramentas necessárias para a manipulação de recursos multimidiáticos, dando um enfoque operacional. Ela engloba desde os fundamentos básicos como o estudo dos hardwares e o ambiente para o seu desenvolvimento, até estruturas mais complexas como realidade aumentada, interface e navegação de interação. Devido a sua armação diversificada, ela possui alcance em importantes áreas da computação, dando enfoque à visualização de dados e realidade virtual.\citep{9gag}

\section{Relevância}

O curso se originou pelo interesse dos alunos e docentes na criação de mundos virtuais e nas aplicações da multimídia nesses mundos. Atualmente é ministrado pelo professor Giordano e, em pós-graduação, recebe o nome de "Realidade virtual e Multimídia", já na graduação é conhecida como "Introdução à Multimídia", e dá ênfase a implementação de mundos virtuais como suporte a aplicações multimídia. Tem como objetivos desenvolver a capacidade de propor, criar e avaliar ambientes virtuais, visando aplicações em educação, visualização de dados, arquitetura e urbanismo, medicina, entretenimento e etc, provando sua importância não só para o meio digital isolado, mas também para o  entrelaçamento da computação em outras áreas.\citep{grade}\\
Ao final da disciplina, se espera que o estudante tenha a capacidade de entender e manipular o conceito de interatividade, fundindo com os conceitos de computação gráfica, levando sempre em consideração as necessidades dos usuários. \cite{obj} 



\section{Relação com outras disciplinas}
\begin{table}[h!]
\begin{tabular}{|l|l|}
\hline
\textbf{IF669-Introdução à Programação} & \begin{tabular}[c]{@{}l@{}}A integração de mídias requer conhecimentos\\  na área de programação de softwares, uma vez\\ que necessita da programação de comandos essenciais\\  para o seu funcionamento.\end{tabular} \\ \hline
\textbf{IF756-Autoria Multimídia}       & \begin{tabular}[c]{@{}l@{}}É um sistema de multimídia voltado à\\  autoria de cursos de multimídia.\end{tabular}                                                                                                       \\ \hline
\end{tabular}
\end{table}


\bibliographystyle{IEEEtran}
\bibliography{rgals}

\end{document}
