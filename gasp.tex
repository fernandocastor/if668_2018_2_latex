\documentclass{article}
\usepackage[utf8]{inputenc}

\title{IF672 - Algoritmos e Estruturas de Dados}
\author{Guilherme Afonso Silva Pereira}
\date{Outubro 2018}

\usepackage{natbib}
\usepackage{graphicx}
\usepackage[brazil]{babel}
\usepackage[utf8]{inputenc}
\usepackage[T1]{fontenc}
\usepackage[normalem]{ulem}
\useunder{\uline}{\ul}{}

\begin{document}

\maketitle

\section{Introdução}
A disciplina de Algoritmos e estrutura de dados aborda algoritmos de busca e ordenação em estruturas de dados, sendo elas estáticas e dinâmicas, homogêneas e heterogêneas, bem como o conceito intuitivo de complexidade de algoritmo\cite{sitedisciplina}\cite{sitedisciplina2}.De certa forma pode-se dizer que a disciplina em questão está inserida em mais de uma grande área da computação, sendo elas a computação básica e a matemática computacional\cite{sitesubareas}.

\section{Relevância}
Algoritmos e estruturas de dados é muito importante para a graduação em Ciência da computação, visto que a disciplina trata de um dos principais conceitos do curso, que é a resolução de problemas através do uso de algoritmos e também estruturar e analisar dados se utilizando de algoritmos próprios para isso. A seguir veja alguns pontos positivos e negativos sobre a disciplina.
\newline
\newline
Pontos positivos:
\begin{itemize}
   \item Aborda principais tópicos de estruturas de dados
   \item Trata de complexidade de algoritmos
   \item Dá aos alunos boa base para as disciplinas seguintes do curso
   \item Tem muita importância no curso de forma geral
 \end{itemize}
 \noindent
Pontos negativos:
\begin{itemize}
   \item Nível de dificuldade alto em comparação a disciplina que a precede
   \item Forma de avaliação com uma só prova
\end{itemize}
\vspace{1.2cm}
\section{Relação com outras disciplinas}
\begin{table}[ht]
\begin{tabular}{|c|l|}
\hline
\multicolumn{1}{|l|}{\textbf{Disciplina}}                                             & \textbf{Relação com Algoritmos e E.D.}                                                                                                                                                                                                                                                                                    \\ \hline
\begin{tabular}[c]{@{}c@{}}IF669\\ Introdução a \\ programação\cite{Wiki}\end{tabular}           & \begin{tabular}[c]{@{}l@{}}Tem relação direta com a disciplina de Algoritmos\\ visto que é um pré-requisito da mesma. É nessa \\ disciplina que o aluno aprende as noções básicas,\\ e um pouco das intermediarias sobre programação.\end{tabular}                           \\ \hline
\begin{tabular}[c]{@{}c@{}}IF689\\ Informática teórica\cite{Wiki}\end{tabular}                   & \begin{tabular}[c]{@{}l@{}}Ao contrário da anterior, Algoritmos é pré-requisito \\ para essa disciplina. A principal relação entre as duas\\ é que ambas estudam a complexidade dos algoritmos, \\ sendo que essa disciplina busca definir qual modelo \\ computacional utilizar para cada tipo de problema.\end{tabular} \\ \hline
\begin{tabular}[c]{@{}c@{}}IF685\\ Gerenciamento de dados\\ e informação\cite{Wiki}\end{tabular} & \begin{tabular}[c]{@{}l@{}}Tem relação por se tratar de uma cadeira de banco \\ de dados, logo a disciplina de algoritmos serve \\ como base para que o aluno tenha noção de estruturas \\ de dados, bem como métodos eficazes de varredura e \\ busca nos mesmos.\end{tabular}                                           \\ \hline
\end{tabular}
\end{table}

\section{Referências - Bibliografia da disciplina}
\begin{enumerate}
    \item Anany Levitin. Introduction to the design and analysis of algorithms (3rd ed). Addison Wesley, 2011.
    \item Clifford Shaffer. Data Structures and Algorithm Analysis . Dover Publications, 2013
    \item Sanjoy Dasgupta, Christos Papadimitriou, Umesh Vazirani. Algoritmos. McGraw Hill, 2009
    \item Thomas H. Cormen, Charles E. Leiserson, Ronald L. Rivest and Clifford Stein. Introduction to Algorithms. MIT Press, 2009.
\end{enumerate}

\bibliographystyle{plain}
\bibliography{gasp.bib}
\end{document}
