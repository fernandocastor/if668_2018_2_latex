\documentclass[10pt]{extarticle}
\usepackage[utf8]{inputenc}
\usepackage[brazil]{babel}
\usepackage{hyperref}
% \usepackage{booktabs}


\title{ET586 - Estatística e Probabilidade para Computação}
\author{Antônio Barros da Silva Netto}
%s\date{October 2018}

\usepackage{natbib}
\usepackage{graphicx}

\begin{document}

\maketitle

\section{Introdução}

\hspace{\parindent}Essa disciplina obrigatória que pertence a grande área de "Estatística Computacional", tem como função oferecer uma abordagem básico sobre alguns aspectos estáticos e probabilístico, como por exemplo: "Calculo de probabilidades, Estimativa e Espaço Amostral". Ela é caracterizada pelo estudo e analise dos mais diferentes estilos de conjuntos, assim buscando alternativas mais simples (porém não menos complexas), para os que à estudam, consigam resolver seus desafios.

\begin{figure}[h!]
\centering
\includegraphics[width=0.5\textwidth]{curvagaussiana.png}
\caption{Exemplo de "Curva Gaussiana", estilo de gráfico que mostra a probabilidade de ocorrência de certos eventos}
\url{<https://tinyurl.com/y83bhev7>}
\href{https://tinyurl.com/y83bhev7}{Autor: Thais Monteiro Peres ~ ~ ~ ~ ~ ~ ~ Licensa: Creative Commons Attribution-Share Alike 4.0 International}
\label{fig:curvagaussiana}
\end{figure}

\section{Relevância}

\hspace{\parindent}A cadeira é importante pro currículo de um aluno do Centro de Informática por vários motivos, considera-se que Estática e Probabilidade é importante para introduzir os conceitos de análise matemática, considerando-se que alunos do CIn provavelmente vão para empresas que esperam que saibam como exemplo, analisar projetos ou até mesmo impactos de certos produtos em determinada área e tenham conhecimento de como trabalhar com isto.

A cadeira tem alguns pontos negativos, como focar demais na parte de "Estática", acaba não trabalhando muito com alguns conceitos computacionais e são extremamente teóricas, apenas um projeto ou outro que são focados nessa parte. Assim dando a impressão que se está em um curso próprio de "Estática". Porém ressalvando, a cadeira continua sendo muito importante, apenas são pequenos pontos que incomodam a experiência do aluno.

\section{Relação com outras disciplinas}
\begin{table}[h!]
\centering
\begin{tabular}{|l|l|}
\multicolumn{1}{l}{\textbf{~ ~ ~ ~ ~ ~ ~ ~ ~ ~ ~ ~ ~ CADEIRA}} & \multicolumn{1}{l}{~ ~ ~ ~ ~ ~  ~ ~ ~ ~ ~\textbf{DESCRIÇÃO}}                                                                                                                                       \\ 
\hline
MA026~- CALCULO 1 INTEGRAL E DIFERENCIAL                       & \begin{tabular}[c]{@{}l@{}}"Pré-requisito" da cadeira ET586. Cadeira que \\introduz ao aluno o conceito de cálculos \\avançados na matemática além de análises a \\gráficos.\end{tabular}                         \\ 
\hline
IF797 - OTIMIZAÇÃO                                             & \begin{tabular}[c]{@{}l@{}}Cadeira que necessita do ET586 para cursar. \\Se relacionam pela necessidade de analisar \\gráficos e achar seus extremoscomo exemplo.\end{tabular}                                    \\ 
\hline
IF702 - REDES NEURAIS                                          & \begin{tabular}[c]{@{}l@{}}Cadeira que se relaciona com o ET586, pelo \\estilo conteudista. Sendo a cadeira de Redes \\Neurais, sempre precisando de uma análise \\de dados.\end{tabular}                         \\ 
\hline
IF689 - INFORMÁTICA TEORICA                                    & \begin{tabular}[c]{@{}l@{}}Cadeira que se relaciona com o ET586, pelo \\estilo de problemas apresentados na cadeira, \\como algorítimos mais complexos e~como \\entende-los a partir da matemática.\end{tabular}  \\ 
\hline
IF699 - APRENDIZAGEM DE MÁQUINA                                & \begin{tabular}[c]{@{}l@{}}Cadeira que se relaciona com o ET586, pela \\constante análise de dados e problemas, na \\tentativa de achar um padrão.\end{tabular}                                                   \\ 
\hline
\multicolumn{1}{l}{}                                           & \multicolumn{1}{l}{}                                                                                                                                                                                              \\
\end{tabular}
\end{table}


\cite{1}
\cite{2}
\cite{3}
\bibliographystyle{plain}
\bibliography{references}

\end{document}
