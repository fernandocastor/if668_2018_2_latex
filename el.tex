\documentclass{article}
\usepackage[utf8]{inputenc}

\title{IF678 - Infra-Estrutura de Comunicação}
\author{Erike Lincon Ferreira do Nascimento}
\date{25 de Outubro de 2018}

\usepackage{natbib}
\usepackage{graphicx}

\usepackage[brazil]{babel}
\usepackage[utf8]{inputenc}
\usepackage[T1]{fontenc}

\begin{document}

\maketitle

\section{Introdução}
 Esta disciplina introduz os principais conceitos e aplicações em redes de computadores, serviços web, segurança em redes e sistemas celulares, para citar alguns exemplos. \par
 A discplina tem o objetivo de abordar a comuciação entre humano e máquina assim como máquina e humano, em suas várias formas, com um foco em rede de computadores e a internet; já que está a principal face dessas interações nos dias atuais.
 
 \begin{figure}[h!]
\centering
\includegraphics[scale=0.25]{IEC}
\caption{Exemplo \cite{gregsubamarine}}
\label{fig:IEC}
\end{figure}

 \section{Relevância}
 A disciplina leva o estudante a apreciar uma relação mais íntima com as estruturas de comunicação utilizadas na computação, conhecendo melhor as terminologias tecnicas. \par
 O aprofundamento nesses conhecimentos pode abrir possibilidades ilimitadas do que se é conhecido como possível em algumas áreas do curso; por exemplo, a segurança digital.
 \subsection{Prós}
 \begin{itemize}
     \item Seu conteúdo é de extrema importância para o estudo de redes.
     \item Aborda uma enorme gama de estruturas usadas em comunicação.
     \item As aulas são interativas, com participação dos alunos.
     \item Trata de exemplos reais na abordagem das disciplinas.
 \end{itemize}
 \subsection{Contras}
 \begin{itemize}
     \item Muito apegada a tecnicidades terminológicas.
     \item Aborda muitos conteúdos diferentes de forma rasa.
 \end{itemize}

\section{Relação com outras disciplinas}
\begin{table}[h!]
\begin{tabular}{|l|l|}
\hline
Disiciplina             & Relação com Infra-Estrutura da Comunicação                                            \\ \hline
Introdução a Computação & \begin{tabular}[c]{@{}l@{}}Ambas possuem um enfoque em uma \\ abordagem teórica nas aulas,\\ com vários módulos, onde cada um\\ aborda uma camada diferente das\\ abordagens, um entre as infraestruturas\\ de comunicação, a outra entre as diversas\\ aŕeas da computação. Também se torna\\ presente a apresentação de mini-projetos\\ ao longo do período.\end{tabular} \\ 
\hline
\end{tabular}
\end{table}

\bibliographystyle{plain}
\bibliography{references}
\nocite{james2013redesdecomputadores}
\nocite{james2015redesdecomputadores}
\nocite{andrew2011redesdecomputadores}
\end{document}
