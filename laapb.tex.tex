\documentclass{article}
\usepackage[brazil]{babel}
\usepackage[utf8]{inputenc}
\usepackage{graphicx}

\title{IF668 - Introdução à Biologial Molecular Computacional}
\author{Lukas Asael A. P. Bacelar }
%\date{Outubro 2018}

\usepackage{natbib}
\usepackage{graphicx}


\begin{document}

\maketitle

\section{Introdução}
Introdução à Biologia Molecular Computacional é uma disciplina eletiva dos cursos da área computacional que tem como objetivo introduzir conceitos essenciais da Biologia para a compreensão da área de aplicação, dos problemas práticos de Bio-Informática e Biologia Computacional que envolve a manipulação e análise de dados biológicos e problemas da área, juntamente com abordagens computacionais para a sua solução.

Sendo assim, essa disciplina se insere na área de Bioinformática e é lecionada pela professora Katia Silva Guimarães, Ph.D. em Ciência da Computação pela universidade de Maryland USA.

\section{Relevância}
Essa disciplina aparece no currículo de Ciência da Computação a partir do sexto período, e é de suma importância para a área de Bioinformática juntamente com Introdução à Bio-informática e Biologia Computacional, pois é ela que dá a base da biologia para todas as demais disciplinas que abordam a área da Bioinformática.

    \item Pontos positivos
      \begin{enumerate}
         \item Uma área muito importante inclusive para pesquisa quanto à doenças e mutações em células como o câncer.
         \item Muito atraente para pessoas que gostam de biologia e de computação.
      \end{enumerate}
    \item Pontos negativos
      \begin{enumerate}
         \item A disciplina só está disponível a partir do sexto período o que deixa os alunos ansiosos pelo menos por uma base na área até lá
         \item Existe pouco material de informação sobre a disciplina
      \end{enumerate}



\section{Relação com outras disciplinas}

\begin{table}[h!]
\centering
\label{my-label}
\begin{tabular}{|p{6.7cm}|p{7.7cm}|}
\hline
IN1115 – Introdução à Bio-informática e Biologia Computacional & \begin{tabular}[c]{@{}l@{}}Juntamente com Introdução à Biologia Molecular \\ Computacional, essa disciplina da a Base para \\ Bioinformática, abordando:

\\ -Alinhamento de Sequências e Sequenciamento de \\ DNA

\\-Classificação e Anotação de Sequências Biológicas

\\-Estruturas de Dados Biológicos e Busca em Cade- \\ ias

\\-Transcrição, regulação e expressão gênica 

\\-Dentre outros.\end{tabular}\\ \hline
\end{tabular}
\end{table}

\nocite{Chapman}
\nocite{Joao}
\nocite{Pavel}

\bibliographystyle{plain}
\bibliography{biblio}


\end{document}



