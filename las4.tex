\documentclass[10pt]{article}
\usepackage[utf8]{inputenc}
\usepackage[brazil]{babel}
\usepackage[T1]{fontenc}
\usepackage{indentfirst}

\title{IF674 - Infra-Estrutura de Hardware}
\author{Leonardo dos Anjos Silva}
\date{Outubro de 2018}

\begin{document}

\maketitle

\section{Introdução}

A disciplina de Infra-Estrutura de Hardware tem como objetivo apresentar os principais componentes de um computador, explicando suas funções e funcionamentos. São tratados, neste curso, os componentes: processador, sistema de memória -- memória principal e memória cache --, dispositivos de entrada e saída e barramentos.

Para consolidar o aprendizado, o aluno desta disciplina tem a oportunidade de aplicar os conhecimentos, desenvolvendo um projeto de uma versão simples do componente estudado ou simulando o funcionamento do mesmo, por meio de uma ferramenta de simulação.

Desta forma, este curso permite que o futuro profissional de Ciências da Computação entenda projetos de implementação de computadores e seja capaz, também, de desenvolvê-los.

O curso é ministrado pelo docente Adriano Sarmento e tem como referência bibliográfica oficial os livros: Organização e Projeto de Computadores: A Interface Hardware/Software - 3ª Edição \cite{firstBookPortuguese}, Computer Organization and Design RISC-V Edition - The Hardware Software Interface \cite{firstBookEnglish}  e Arquitetura e Organização de Computadores \cite{secondBook}.

\section{Relevância}

Ao passo que Infra-Estrutura de Hardware estuda o funcionamento dos componentes de um computador, Infra-Estrutura de Software trata dos sistemas de software básico e sistemas operacionais, e Infra-Estrutura de Comunicação explica redes de computadores, o funcionamento da internet e protocolos de comunicação. Em conjunto com estas disciplinas, o curso de Infra-Estrutura de Hardware fornece o conhecimento necessário para a construção de muitos sistemas de computação atuais.

\subsection{Pontos Positivos}

\begin{itemize}
    \item Este curso fornece uma visão mais ampla do funcionamento de um computador.
    \item O aluno desta disciplina tem a oportunidade de aplicar o conhecimento, por meio de ferramentas de simulação ou implementação de um mini-projeto.
    \item A média global do curso envolve provas, listas e projeto, o que possibilita chances maiores de um estudante conseguir aprovação.
    \item O site da disciplina contém diversas informações úteis, como uma planilha de notas e os aquivos das aulas em PDF ou PPT.
\end{itemize}

\subsection{Ponto Negativo}

\begin{itemize}
    \item Sozinha, esta cadeira não é suficiente para entender o funcionamento de um sistema de computação qualquer.
\end{itemize}

\section{Relação com Outras Disciplinas}

Na Tabela \ref{tab:connections}, encontram-se relações entre a disciplina de Infra-Estrutura de Hardware e outras disciplinas do curso de Ciências da Computação.

\begin{table}[!htb]
    \centering
    \begin{tabular}{|l|p{0.5\textwidth}|}
    \hline
    \textbf{\textsc{Disciplina}} & \textbf{\textsc{Relação}} \\ \hline
    IF118 - Organização de Computadores & É equivalente à disciplina de Infra-Estrutura de Hardware no perfil curricular de Ciência da Computação. \\ \hline
    IF669 - Introdução a Programação & É pré-requisito da disciplina. \\ \hline
    IF675 - Sistemas Digitais & É pré-requisito da disciplina. \\ \hline
    IF677 - Infra-Estrutura de Software & Faz parte da tríade que envolve as disciplinas de Infra-Estrutura de Comunicação, Hardware e Software. \\ \hline
    IF678 - Infra-Estrutura de Comunicação & Faz parte da tríade que envolve as disciplinas de Infra-Estrutura de Comunicação, Hardware e Software. \\ \hline
    \end{tabular}
    \caption{Relação com Outras Disciplinas}
    \label{tab:connections}
\end{table}

\bibliographystyle{plain}
\bibliography{las4.bib}

\end{document}
