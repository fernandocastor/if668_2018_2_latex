\documentclass[10pt, a4paper]{article}
\usepackage[utf8]{inputenc}

\usepackage[brazil]{babel}
\usepackage[utf8]{inputenc}
\usepackage[T1]{fontenc}

\title{IF699 - Aprendizagem De Máquina}
\author{Victor Edmond Freire Gaudiot}
%\date

\usepackage{natbib}
\usepackage{graphicx}

\begin{document}

\maketitle

\begin{figure}[h!]
\centering
\includegraphics[scale=0.75]{aa.png}
\caption{Aprendizagem De Máquina}
\label{fig:cérebro de máquina}
\end{figure}

% link da imagem :https://de.wikipedia.org/wiki/Deep_Learning
% licença :wikipedia

\section{Introdução}
~~~~~Ofertada  no primeiro semestre de cada ano, é uma cadeira do 6º período do curso do tipo eletiva, e tendo como professor George Darminton. Ao abordar assuntos tais como: Árvore de decisão, distâncias heterogêneas, avaliação de hipóteses, dentre outros, \citep{siteDisciplina} a disciplina ensina a construir algoritmos capazes de aprender com seus próprios erros e fazer previsões sobre dados.

Ela se insere na área de inteligência artificial, mas aprendizado de máquina se preocupa apenas com o aprendizado indutivo, ou seja, que extrai regras e padrões de grandes conjuntos de dados.\citep{WikiDisciplina}

\section{Relevância}
~~~~~Essa disciplina se encontra no nosso currículo devido às suas diversas aplicações, tais como mecânismo de busca, jogos de estratégia, classificação de sequências de DNA. Aprender a generalizar a partir de experiências. O interesse nesta cadeira também decorre no aumento da acessibilidade de informações, do custo benefício dos processadores, armazenamento de dados e etc.\citep{Site1}

\section{Relação com outras disciplinas}
\begin{table}[h]
\centering
\begin{tabular}{|l|l|} 
\hline
IF702 - Redes Neurais                    & \begin{tabular}[c]{@{}l@{}}É utilizado na aprendizagem de máquina. Utili-\\zado para reconhecer padrões, o que auxilia a\\máquina a aprender\end{tabular}                \\ 
\hline
IF685 - Gerenciamento Dados e Informação & \begin{tabular}[c]{@{}l@{}}Ambos trabalham com os dados de forma confi-\\áveis, compreensíveis e otimizadas.\end{tabular}                                                \\ 
\hline
ET586 - Estatística e Probabalidade      & \begin{tabular}[c]{@{}l@{}}A relação ocorre por serem dois campos\\intimamente ligados. Há aqueles que sugerem\\chamar o campo todo de "ciência de dados".\end{tabular}  \\
\hline
\end{tabular}
\end{table}


\bibliographystyle{plain}
\bibliography{references}
\nocite{Dados}


\end{document}
