\documentclass{article}
\usepackage[utf8]{inputenc}
\usepackage[brazil]{babel}

\title{IF689 -INFORMATICA TEORICA}
\author{André Vasconcelos}
\date{Outubro de 2018}

\usepackage{natbib}
\usepackage{graphicx}
\usepackage{indentfirst}
\begin{document}

\maketitle

\section{Introdução}
Esta disciplina é de cunho \textbf{obrigatório} para os cursos de Ciência da Computação, no 4º Período, e Engenharia da Computação, no 6º Período. Atualmente, esta disciplina é ministrada pelo professor Fred Freitas (CC) e Ruy Queiroz (EC).

A disciplina aborda tópicos relacionados à noção de procedimento efetivo. A exemplo da Máquina de Turing, a qual mostrou que a computação das operações de leitura, poderiam ser satisfeitas por uma máquina que continha uma fita de comprimento ilimitado. Com isso, foi gerado o termo algoritmo, abordado, também, nesta disciplina.

Sua bibliografia básica é composta por: \cite{1}. E sua bibliografia suplementar é composta por: \cite{2} \cite{3} \cite{4} \cite{5}.
    \begin{figure}[h!]
    \centering
    \includegraphics[width=0.4\textwidth]{bletchey.JPG}
    \caption{Máquina de Turing - Licença : \cite{10}}
    \label{fig:universe}
    \end{figure}
    
    
\section{Relevância}
A disciplina tem como principal objetivo capacitar os alunos na compreensão de como se funciona a computação e os limites da mesma.

Ao final do semestre, o aluno terá um conhecimento amplo sobre computabilidade, decidibilidade, as linguagens decididas por autômato, a tese de Church-Turing, funções computáveis e problemas intratáveis. Cognições chave para um graduante da área de computação.
\section{Relação com outras disciplinas}

    \begin{table}[h]
    \centering
    \begin{tabular}{|p{7cm}|p{6cm}|}
    \hline
    IF774 - Complexidade Descritiva & Aprofunda os conhecimentos sobre a complexidade, assunto abordado pela Informática Teórica.   \\ \hline
    IF776 - Topicos Avançados em  em Informática Teórica & É o estudo mais detalhado acerca da Informática Teórica. \\ \hline
    IF770- Teoria Dos Modelos & Investiga-se o que pode ser concluído de alguns objetos matemáticos pré-existentes, algumas operações e/ou relações entre estes objetos, e alguns axiomas. \\ \hline
    IF772 - Lambda Cálculo Teoria Tipos &  Estuda funções recursivas computáveis, no que se refere a teoria da computabilidade, e fenômenos relacionados, como variáveis ligadas e substituição. \\ \hline
    \end{tabular}
    \end{table}

\bibliographystyle{plain}
\bibliography{alpvj}

\end{document}