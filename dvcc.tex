\documentclass[10pt,a4paper]{article}

\usepackage[brazil]{babel}
\usepackage[utf8]{inputenc}
\usepackage[T1]{fontenc}

\usepackage[a4paper,top=2cm,bottom=2cm,left=3cm,right=3cm,marginparwidth=1.75cm]{geometry}


\usepackage[colorinlistoftodos]{todonotes}
\usepackage[colorlinks=true, allcolors=blue]{hyperref}

\title{IF680 - Processamento Gráfico}
\author{Daniel Victor Cintra Cavalcante}

\begin{document}

\maketitle

\section{Introdução}
Processamento Gráfico estuda os processos ou técnicas computacionais que envolvem modelos geométricos e imagens digitais. Os projetos nesta área de pesquisa envolvem a transformação de dados geométricos em imagens, a transformação de imagens em dados diversos, o estudo de formas eficientes para apresentar visualmente grandes volumes de dados, o desenvolvimento de algoritmos para auxiliar na descoberta de estruturas de interesse presentes em imagens, o estudo de formas de representação e manipulação de modelos geométricos tridimensionais, além de técnicas de visualização e interação dentro de sistemas de Realidade Virtual e Aumentada. As aplicações das pesquisas contemplam a área de saúde, educação, entretenimento, bioinformática, jogos, entre outras. Exemplos de aplicações desenvolvidas no PPgSI são: sistemas de auxílio ao diagnóstico, sistemas de treinamento em saúde, jogos sérios para reabilitação, análise morfológica de seres vivos e reconhecimento de língua de sinais.



\section{Relevância}
Processamento gráfico é uma área q trabalha com imagens, logo, com a facilitação de meios de obtenção de imagens nos ultimos anos, como câmeras fotográficas e celulares, houve uma ploriferação do material com qual a área se envolve, e assim, maior demanda e recursos fazendo que a área ganhasse grande relevância no cenário atual.\newline
\begin{figure}[h!]
    \centering
    \includegraphics[width=0.4\textwidth]{Graphics_Processing_Unit.JPG}
    \caption{GPU AMD - Licença : \cite{img}}
    \label{fig:universe}
    \end{figure}
\newline
Além disso, com melhor desempenho dos processadores gráficos, a área ganhou ainda maior abrangência, com a possibilidade de melhor representação de dados geometricos em computadores, áreas como engenharia e arquitetura fazem bom uso dos recursos aprendidos no curso.
\newpage

\section{Relação com outras áreas}
Processamento gráfico é dividida em três grandes áreas (ou linhasde pesquisa), que muitas vezes se combinam para se atingir o objetivo desejado, essas áreas são:

\begin{table}[h]

\centering

\label{my-label}

\begin{tabular}{|p{5cm}|p{7cm}|}

\hline

Processamento de imagens  & \begin{tabular}[c]{@{}p{7cm}@{}}Visa a
obtenção de informações da imagem para produção de dados a respeito da mesma ou modificação da imagem.\end{tabular}\\ \hline
Computação Gráfica  & \begin{tabular}[c]{@{}p{7cm}@{}} sintetiza imagens a partir de um
conjunto de dados, buscando o "melhor resultado" com o menos custo computacional. \end{tabular}\\ \hline
Visão computacional  & \begin{tabular}[c]{@{}p{7cm}@{}}Desenvolve teoria e tecnologia para a construção de sistemas artificiais que obtém informação de imagens ou quaisquer dados multi-dimensionais. \end{tabular}\\ 
\hline
\end{tabular}
\end{table}

\bibliographystyle{plain}
\bibliography{references}
\nocite{CG}
\nocite{CS}
\nocite{OP}
\end{document}
