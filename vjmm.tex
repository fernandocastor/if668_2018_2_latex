\documentclass[10pt, a4paper]{article}

\usepackage[brazil]{babel}
\usepackage[utf8]{inputenc}
\usepackage[T1]{fontenc}

\usepackage[left=3cm,right=3cm,bottom=1,2cm]{geometry}

\usepackage{amsmath}
\usepackage{graphicx}
\usepackage[colorinlistoftodos]{todonotes}
\usepackage[colorlinks=true, allcolors=blue]{hyperref}

\title{IF754- Computação Musical e Processamento de Som}
\author{Vlademir José Montenegro de Melo}

\usepackage{natbib}
\usepackage{graphicx}

\begin{document}

\maketitle

\begin{figure}[h!]
\centering
\includegraphics[scale=1.3]{foto1}
\label{fig:foto1}
\end{figure}

\section{Introdução}
Esta disciplina tem como foco passar uma visão geral sobre a computação dentro do mundo multimídia, possuindo como principais matérias relativas ao mundo visual e auditivo, respectivamente, a computação gráfica e a computação musical. Além disso, também oferece aos alunos a possibilidade de complementar seus conhecimentos relativos à natureza da forma sonora, aos algoritmos para a síntese e processamento de sons digitais, e às técnicas de representação e manipulação de informações  musicais, incluindo wave, MIDI, MP3, RealAudio, etc.

\section{Relevância}
Essa disciplina pertence ao nosso currículo(\citep{Curriculo}) por englobar um artifício de grande utilidade, a aprendizagem de como "manipular" o som. A cadeira, aparentemente, não possui nenhum ponto negativo, pois o aluno não precisa de nenhum conhecimento profundo acerca da área musical ou gráfica, e ao entrar nela, qualquer indivíduo terá completa condição de se desenvolver nas áreas envolvidas. Para mais informações, a disciplina possui os sites \citep{Site_oficial} e \citep{Site_secundario}.

\section{Relação com outras disciplinas}


\begin{table}[h]
\begin{tabular}{|l|l|}
\hline
IF672 - Algoritmos e Estruturas de Dados & \begin{tabular}[c]{@{}l@{}}Essa disciplina tem como principal função a de\\ auxiliar o aluno para que o mesmo possa atingir\\ um maior nível de otimização dentro de um código.\end{tabular}                                                                         \\ \hline
IF101 - Linguagens de Programação 3      & \begin{tabular}[c]{@{}l@{}}Essa cadeira tem como contribuição promover o\\ conhecimento na área da programação, tornando-se\\ possível construir e fazer utilização de programas\\ voltados para qualquer que seja a área, inclusive na \\ área musical.\end{tabular} \\ \hline
\end{tabular}
\end{table}

\bibliographystyle{plain}
\bibliography{references}
\end{document}
