\documentclass[a4paper]{article}

\usepackage[brazil]{babel}
\usepackage[utf8x]{inputenc}
\usepackage[T1]{fontenc}

\usepackage[a4paper,top=3cm,bottom=2cm,left=3cm,right=3cm,marginparwidth=1.75cm]{geometry}

\usepackage{graphicx}
\usepackage[colorinlistoftodos]{todonotes}
\usepackage[colorlinks=true, allcolors=blue]{hyperref}
\usepackage{booktabs}
\usepackage{adjustbox}

\title{IF704 - Processamento de Linguagem Natural}
\author{Erick de Almeida Oliveira Riso}
\date{25 de outubro de 2018}

\begin{document}
\maketitle

\section{Introdução}

O Processamento de Linguagem Natural (do inglês Natural Language Processing, às vezes também referida como PLN) é uma disciplina eletiva do curso de graduação de Ciência da Computação da Universidade Federal de Pernambuco (UFPE), ministrada pelo professor Hansenclever Bassani\footnote{BASSANI, Hansenclever. \textit{Página oficial da disciplina}. Disponível em: <http://www.cin.ufpe.br/~hfb/pln/index.html>. Acesso em: 23 de outubro de 2018}. É considerada uma subárea da Inteligência Artificial, e tem como objetivo tornar possível a interpretação das línguas humanas naturais pelos sistemas computadorizados. O Processamento de Linguagem Natural incorpora várias técnicas oriundas de disciplinas diferentes, que vão desde a Linguística, através de itens como a semântica, sintaxe e lexicologia, até a Estatística e métodos de aprendizagem de máquina.

\section{Relevância}

A comunicação humana geralmente envolve fatores muito complexos, como contexto e especificidades regionais, e está sujeita à ambiguidades. Os computadores, todavia, geralmente trabalham com dados estruturados e linguagem precisa. Conciliar esse problema é um dos desafios do Processamento de Linguagem Natural.
Grandes empresas têm dado muita importância ao ramo, já que esta área pode auxiliar a resolução de muitos problemas hodiernos da computação, como a análise de Big Data e o aperfeiçoamento de sistemas de aprendizado de máquina. Muitos analistas apostam, inclusive, que o Processamento de Linguagem Natural seja parte integrante do que virá a ser ``a grande novidade'' da tecnologia da informação.\footnote{DEANGELIS, Stephen F. \textit{The Growing Importance of Natural Language Processing}. Disponível em: <https://www.wired.com/insights/2014/02/growing-importance-natural-language-processing/> (em inglês). Acesso em: 25 de outubro de 2018}

Hoje, muitas aplicações utilizam processamento de linguagem natural, e entre elas podemos destacar:
\begin{itemize}
\item Tradução automática de textos
\item Reconhecimento de voz
\item Fitragem de SPAM
\item Reconhecimento ótico de caracteres
\item Análise de sentimentos
\item Sumarização de documentos
\item Correção ortográfica
\item Chatbots e assistentes pessoais
\end{itemize}
\pagebreak
\subsection{Pontos positivos e negativos}
Entre os pontos positivos da disciplina, evidencia-se:
\begin{itemize}
\item Área desfruta de relevância para grandes empresas
\item Grande interdisciplinaridade com outras áreas, como Inteligência Artificial, Aprendizado de Máquina, Estatística, Linguística e até mesmo Psicologia
\item Liberdade para o estudante poder testar e explorar, já que não há apenas uma única forma de se resolver um determinado problema
\end{itemize}
Já entre os pontos negativos, é possível salientar:
\begin{itemize}
\item Pouco material de estudo disponibilizado em língua portuguesa
\item Se comparado à outras línguas, como o inglês, o português ainda não tem tanto destaque enquanto objeto de estudo da disciplina, o que se reflete em menos ferramentas para processamento e análise de tal língua.
\end{itemize}




\section{Relação com outras disciplinas}

\begin{table}[h]
\begin{tabular}{|p{5cm}|p{9,2cm}|}
\hline
\textbf{Disciplinas}                                & \textbf{Relação}                                                                                                                                                                                                                                                                      \\ \hline
Processamento de Voz (IF759)                        & Um dos ramos do Processamento de Linguagem Natural é o Processamento de Voz. É esperável, portanto, que esta disciplina tenha o IF704 como pré-requisito.                                                                                                                             \\ \hline
Sistemas Inteligentes (IF684)                       & Uma importante demanda na criação de sistemas inteligentes é a de interagir e tentar compreender o interesse dos usuários. Quando esta tarefa depende da manipulação de linguagem humana, o Processamento de Linguagem Natural pode ser um aliado.                        \\ \hline
Introdução à Programação (IF669)                    & Visto que o Processamento de Linguagem Natural lida com a criação de programas e sistemas para solucionar problemas de interpretação de linguagem humana, habilidades em programação são imprescindíveis.                                                                             \\ \hline
Estatística e Probabilidade para Computação (ET586) & A Inteligência Artificial se vale de muitos conceitos da estatística, como a probabilidade, para a tomada de decisões. No caso específico do Processamento de Linguagem Natural, ela pode ser utilizada para definir o contexto de uma expressão em um texto específico, por exemplo. \\ \hline
Álgebra Vetorial e Linear (MA531)                   & Muitas noções da álgebra linear e vetorial são utilizadas em modelos para processamento de linguagem natural. Por exemplo, o espaço semântico, ferramenta empregada para representar o significado das palavras, recorre-se aos conceitos de espaço vetorial e vetor.              \\ \hline
\end{tabular}
\end{table}

\nocite{slp}
\nocite{iir}
\nocite{nlpp}

\bibliographystyle{plain}
\bibliography{eaor}

\end{document}