\documentclass{article}
\usepackage[utf8]{inputenc}
\usepackage[brazil]{babel}
\usepackage[T1]{fontenc}
\title{IF682 - ENGENHARIA SOFTWARE E SISTEMAS}
\author{Marcos Prysthon}
\date{Outubro 2018}

\usepackage{natbib}
\usepackage{graphicx}

\begin{document}

\maketitle

\section{Introdução}
Engenharia de Software (IF682) é a cadeira focada em desensolvimento de software, mas, além disso, ela também é focada em trabalho em conjunto. Os processos envolvidos desde a elaboração do software desejado, desde sua base até sua versão final.

Portanto, não se trata apenas de programação, uma atividade que pode ser desenvolvida de forma independente de outras pessoas, mas de atividades que requerem trabalho em equipe e capacidade de comunicação.

A disciplina se aplica em toda área onde há tecnologia associada à softwares, atuando na montagem em si de um certo software, nas relações interpessoais do grupo que está desenvolvendo um certo programa, entre outras áreas de atuação.

A cadeira utiliza a bibliografia o livro base\cite{IanSommerville} e os textos complementares \cite{CarloGhezzi} \cite{AlainAbran} \cite{ShariPfleeger}

\section{Relevância}
A matéria têm diversas áreas de atuação, contando com a criação de programas, aplicativos, páginas na web, entre várias outras aplicações.

Para tal, a cadeira apresenta desde os fundamentos que tornam essa essas produções possíveis até a elaboração, otimização e manutenção.

Logo, a cadeira apresenta enorme relevância tanto na área de programação, quanto na área de trabalho entre alunos associados entre si. Desse jeito, a cadeira tenta lessionar competências sobre programação e trabalho em conjunto.

\begin{enumerate}
    \item Pontos positivos da disciplina:
\begin{enumerate}
    \item Disciplina com grande aplicabilidade na área de trabalho
    \item Grande demanda de programadores com abilidades fornecidas nessa disciplina
    \item Renovação e criação de softwares desejados 
    \end{enumerate}
\end{enumerate}


\section{Relação com outras disciplinas}
A disciplina IF682 se relaciona com outras disciplinas da grade curricular de Ciência da Computação, elas são:
\begin{table}[h]
\centering
\small
\begin{tabular}{11}
\hline
Disciplinas relacionadas: & Relações com IF682: \\
Logica pra Computação (IF673) & \begin{tabular}[c]{@{}l@{}}Essa disciplina,
apresentada no segundo período,\\ se relaciona fortemente com a engenharia
de software\\ por proporcionar uma visão lógica, intuitiva e\\ rápida para
resolução de problemas e elaboração de algoritmos.\end{tabular} \\

Matemática Discreta pra Computação (IF670) & \begin{tabular}[c]{@{}l@{}}A
disciplina IF670 é pré requisito direto da disciplina IF673,\\  dando a base
e as competências para o desenvolvimento do\\  raciocínio lógico e eficiência
para a elaboração de algoritmos\\  e softwares.\end{tabular} \\

Algorítimos e Estrutura de Dados (IF672) & \begin{tabular}[c]{@{}l@{}}Disciplina
apresentada no segundo período, com objetivo de\\  ensinar a construir e escrever
programas mais elaborados e\\  com melhor eficiência, se relaciona diretamente com
um dos\\  principais objetivos da Engenharia de Software, que é pensar,\\  elaborar,
desenvolver e programar projetos com destino ao\\  mercado e com grande eficiência.\end{tabular} \\

Introdução à Programação (IF669) & \begin{tabular}[c]{@{}l@{}}Base de toda técnica
de programação apresentada no curso, a\\  disciplina IF669 ensina os conceitos
básicos da programação,\\  sendo pré requisito direto da disciplina IF672. Logo,
sem os\\ conhecimentos apresentados na IF669 seria impossível progredir\\  na
grade curricular do curso e, ainda mais importante,não seria\\ possível fazer
simples programas e entende-los, impossibilitando\\ a criação de programas
complexos e desqualificando o aluno em\\  relação ao mercado.\end{tabular}
\end{tabular}
\end{table}


\bibliographystyle{plain}
\bibliography{references}
\cite{IanSommerville}
\cite{CarloGhezzi}
\cite{AlainAbran}
\cite{ShariPfleeger}
\end{document}
