\documentclass[10pt]{article}
\usepackage[brazil]{babel}
\usepackage[utf8]{inputenc}
\usepackage{amsmath,amssymb}
\usepackage{natbib}
\usepackage{parskip}
\usepackage{graphicx}
\usepackage{cite}


% Margens
\usepackage[top=2.5cm, left=3cm, right=3cm, bottom=4.0cm]{geometry}

% Get larger line spacing in table
\newcommand{\tablespace}{\\[1.25mm]}
\newcommand\Tstrut{\rule{0pt}{2.6ex}}         % = `top' strut
\newcommand\tstrut{\rule{0pt}{2.0ex}}         % = `top' strut
\newcommand\Bstrut{\rule[-0.9ex]{0pt}{0pt}}   % = `bottom' strut

%%%%%%%%%%%%%%%%%
%     Titulo    %
%%%%%%%%%%%%%%%%%
\title{IF687 - Introdução a Multimídia}
\author{David Santos}
\date{\today}

\begin{document}
\maketitle

%%%%%%%%%%%%%%%%%
 %  Introdução %
%%%%%%%%%%%%%%%%%
\section{Introdução}
A cadeira de introdução à multimídia surgiu pelo interesse de alunos e professores na criação de mundos virtuais e de aplicações de multimídia, também confrontado com a escassez de material do assunto em português. Ela é uma disciplina obrigatória, faz parte da grade curricular do 5º período do curso de Ciência da Computação e atualmente tem como docente o professor Giordano Ribeiro.            
A matéria, formalmente adere alguns nomes dependendo da forma que é lecionada e aprendida. Quando ministrada ao nível de pós-graduação chama-se "Realidade Virtual e Multimídia"(IF124). Ao nível de graduação chama-se "Introdução à Multimídia".

\begin{figure}[h!]
\centering
\includegraphics[scale=1]{RedesSociais.jpg}
\caption{Redes Sociais}
\label{fig:RedesSociais}
\end{figure}
% imagem 1 : https://pixabay.com/pt/meios-de-comunica%C3%A7%C3%A3o-sociais-ajuda-1432937/
% Grátis para uso comercial 
% Atribuição não requerida

%%%%%%%%%%%%%%%%%
%   Relevância   %
%%%%%%%%%%%%%%%%%

\section{Relevância}
Existe um grau de interesse dentro do mercado de trabalho em várias áreas que abordam e abrangem a disciplina como por exemplo: A indústria criativa, tecnologia digital, realidade virtual, realidade aumentada, música, jogos, criatividade computacional, com algumas intersecções entre as mesmas. O Objetivo fundamental da disciplina é trazer o interesse dessas áreas para os alunos. Há também um foco teórico e prático para detectar o interesse dos alunos.

\subsection{Pontos Positivos}
Os assuntos abordados na disciplina geralmente chamam a atenção dos alunos por serem coisas do cotidiano que os mesmos estão habituados a realizar como: Jogos, música.  

\subsection{Pontos Negativos}
A carga horário da disciplina é curta, ou seja, diversos assuntos que poderiam ser abordados, deixam de ser explanados devido ao curto espaço de tempo existente.


\section{Relação com outras disciplinas}

\begin{table}[h]
\centering
\label{my-label}
\begin{tabular}{|l|l|}
\hline
IF755 - Realidade Virtual & \begin{tabular}[c]{@{}l@{}}  A Realidade Virtual é uma disciplina e também\\ uma tecnologia é uma 
vem se mostrando promissora.\\ A cadeira é voltada para pessoas interessadas em\\ computação gráfica, processamento de imagens ou visão \\computacional.   \end{tabular}                                                                                                                                          \\ \hline

IF124 - Realidade Virtual e Multimídia  & \begin{tabular}[c]{@{}l@{}}  O objetivo da disciplina é investigar o estado da arte na\\ área de interação com múltiplos dispositivos buscando\\ identificar abordagens inovadoras e métodos eficientes\\ utilizados para este tipo de interação pela comunidade\\ cientifica.\end{tabular}                                                                                                   
                                                     \\ \hline

\end{tabular}
\end{table}

\bibliographystyle{plain}
\bibliography{references}
\citep{1,2,3}

\end{document}
