\documentclass[10pt, a4paper]{article}
\usepackage[brazil]{babel}
\usepackage[utf8x]{inputenc}
\usepackage[T1]{fontenc}

\usepackage[a4paper,top=3cm,bottom=2cm,left=3cm,right=3cm,marginparwidth=1.75cm]{geometry}

\usepackage{amsmath}
\usepackage{graphicx}
\usepackage[colorinlistoftodos]{todonotes}
\usepackage[colorlinks=true, allcolors=blue]{hyperref}

\title{IF803 - Introdução a Biologia Molecular Computacional}
\author{Matheus Isidoro Gomes Batista}
\date{25 de outubro de 2018}

\begin{document}

\maketitle

\section{Introdução}
\paragraph{}A área da computação promoveu uma série de progressões em todas as áreas do conhecimento, desde a matemática com os algoritmos computacionais que permitiam a resolução de gigantescas equações, matrizes e afins, até a área da psicologia com o uso de IAs para análise de grandes parcelas de dados que permitem uma série de avanços no diagnostico rápido de grande número de pessoas, com a biologia não seria diferente e a computação vem sendo usada com afinco para resolução de problemas dentro da ciência que se dedica ao estudo da vida.
\paragraph{}A disciplina Introdução a Biologia Molecular Computacional (IF803) surge assim para introduzir o estudante da ciência da computação a conceitos da Biologia que são essenciais para a correta compreensão de como a computação pode influir na área, assim como, ela guia através de problemas práticos da Bioinformática e da Biologia Computacional demonstrando a resolução e diferentes soluções envolvendo a computação. Ela é lecionada pela professora Katia Silva Guimarães (Ph.D. em Ciência da Computação pela universidade de Maryland nos Estados Unidos).

\section{Relevância}
\paragraph{}Como ela é a responsável pela introdução a todos os conceitos extremamente importantes para o desenvolvimento da Bioinformática e da Biologia Computacional, ela as precede e surge pela primeira vez a partir do sexto período como uma matéria eletiva.

\paragraph{}Pontos positivos
\begin{enumerate}
\item Ela integra a computação e a biologia, dando oportunidades tanto para a ciência desenvolver-se intercalada entre as matérias quanto para as pessoas que demonstram interesse por ambas as áreas.
\item É usada como introdução para as áreas da Bioinformática e Biologia Computacional que por sua vez são utilizadas em pesquisa de doenças e análise da formação genética (A título de exemplo).
\end{enumerate}

\paragraph{}Pontos negativos
\begin{enumerate}
    \item Pouquíssima parcela de informações quanto a disciplina.
    \item Demora a surgir como eletiva e, inclusive, é pouco conhecida pelos alunos do próprio CIn.
\end{enumerate}
\newpage
\begin{table}[h]

\centering

\label{my-label}

\begin{tabular}{|p{7cm}|p{7cm}|}

\hline

IN1115 – Introdução à Bioinformática e Biologia Computacional  & \begin{tabular}[c]{@{}p{7cm}@{}}Também funciona como uma disciplina introdutória, ela forma toda a estrutura essencial para a Bioinformática e Biologia
Computacional, e explica:
\\- Alinhamento de Sequências de DNA.

\\- Estruturas de Dados Biológicos e Busca em \\Cadeias.

\\- Introdução à Biologia celular.

\\- Famílias de Proteínas e Predição de \\Estruturas.\end{tabular}\\ 
\hline
\end{tabular}
\end{table}

\bibliography{main}
\bibliographystyle{plain}
\nocite{SiteCadeira}
\nocite{CM:ComputationalBiology}
\nocite{CM:BioinformaticsAlgorithms}

\end{document}


