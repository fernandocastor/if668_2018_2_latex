\documentclass{article}
\usepackage[utf8]{inputenc}
\usepackage[brazil]{babel}


\title{IF755 - Realidade Virtual}
\author{Lucas Rodrigues}
\date{Outubro 2018}

\usepackage{natbib}
\usepackage{graphicx}

\begin{document}

\maketitle

\section{Introdução}
A realidade virtual é uma área que estuda formas de interar o usuário à maquina, através de recursos gráficos 3D ou imagens 360º, com o objetivo de criar uma imersão do usuário em determinado ambiente virtual. A disciplina estuda a linguagem CUDA, desenvolvida pela NVIDIA, utilizada como meio de programação para GPU. A disciplina é recomendada para pessoas interessadas em computação gráfica, processamento de imagens ou visão computacional.

\begin{figure}[h!]
\centering
\includegraphics[scale=0.85]{virtual-reality-3368729_960_720}
\caption{Rapaz utilizando um óculos de realidade virtual. \citep{foto}} 
\label{fig:virtual-reality-3368729_960_720}
\end{figure}

\section{Relevância}
Muitos pensam que a realidade virtual é uma área utilizada apenas para fins de entretenimento. Porém, a realidade aumentada pode trazer avanços da área de saúde, por exemplo. É possível, por exemplo, realizar cirurgias virtuais com alto grau de realismo e, dessa forma, especializar novos estudantes de medicina no que tange à realização de cirurgias. Além disso, a realidade virtual pode trazer avanços na área educacional: a empresa Lifeliqe, em parceria com a HTC, criou uma versão do seu aplicativo, que permite estudar animais, plantas, biologia molecular e outras coisas, para o óculos VR HTC Vive, permitindo que estudantes imergam em realidades que não são possíveis de serem inseridas dentro de uma sala de aula. Essa área de estudo também traz utilidades para os que trabalham com marketing: é possivel, por exemplo, desenvolver um aplicativo que permita fazer uma tour virtual por imóveis que ainda não foram construídos. \citep{casamais}

\section{Relação com outras disciplinas}


% ######## init table ########
\begin{table}[h]
 \centering
% distancia entre a linha e o texto
 {\renewcommand\arraystretch{1.25}

 \begin{tabular}{ l l }
  \cline{1-1}\cline{2-2}  
    \multicolumn{1}{|p{3.850cm}|}{Disciplinas \centering } &
    \multicolumn{1}{p{4.217cm}|}{Relações \centering }
  \\  
  \cline{1-1}\cline{2-2}  
    \multicolumn{1}{|p{3.850cm}|}{IF681 - Interfaces Usuário-Máquina  \centering } & \multicolumn{1}{p{4.217cm}|}{A Realidade Virtual, ao captar os movimentos do corpo do usuário (como braços e pernas), gera a maior interação conhecida, até hoje, entre o homem e a máquina. Essa interação passa a ser pensada e desenvolvida em um plano tridimensional, por isso é necessário conhecimentos mais básicos de interfaces usuário-máquina. \citep{grv_pucrs}}
  \\  
  \cline{1-1}\cline{2-2}  
    \multicolumn{1}{|p{3.850cm}|}{IF687 - Introdução à Multimídia. \centering } &
    \multicolumn{1}{p{4.217cm}|}{ Essa disciplina estuda animações, som e vídeo no computador, a possibilidade de integração entre mídias e elementos da realidade virtual, servindo como base para a cadeira de realidade virtual. \citep{ccufpe}}
  \\  
  \hline

 \end{tabular} }
\end{table}



\bibliographystyle{plain}
\bibliography{lrsv}
\end{document}
