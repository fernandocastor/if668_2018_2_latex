\documentclass{article}
\usepackage[brazilian]{babel}
\usepackage[utf8]{inputenc}
\usepackage[T1]{fontenc}
\usepackage{hyperref}
\usepackage{natbib}
\usepackage{graphicx}
\bibliographystyle{abbrv}

\title{IF738- REDES DE COMPUTADORES}
\author{Bruno Nunes de Mesquita }
\date{October 2018}

\begin{document}

\maketitle

\section{Introdução}
A disciplina que hoje abrange as antigas REDES DE COMPUTADORES 1 e REDES DE COMPUTADORES 2 tem por objetivo fornecer ao aluno uma visão global do estado da arte na área de redes de computadores e telecomunicações. Há alguns anos a disciplina eletiva é lecionada pelo professor Djamel Sadok, e tem em sua ementa alguns temas como: \cite{sitedadisciplina}Modelo de Referência OSI, Camada Física (técnicas de transmissão analógica e digital), Técnicas de Multiplexação FDM e TDM e Rede Digital de Serviços Integrados.

As redes de computadores se encontram na subárea da computação de Tecnologias e Sistemas de Computação de acordo com a \cite{sitedaufmg}PROPOSTA DE MUDANÇA DE NOME E CLASSIFICAÇÃO DA CIÊNCIA DA COMPUTAÇÃO EM SUBÁREAS de 1997.

\begin{figure}[!htb]
      \centering
      \includegraphics[scale=0.93]{rededepc.jpg}
      \caption{Rede de computadores. Imagem de uso comum.}
      \label{fig:rede de computadores}
\end{figure}

\section{Relevância}
\cite{sitedaunip}A disciplina de Redes de Computadores é importante na formação do profissional de Ciência da Computação pois trabalha o conhecimento fundamental para o domínio do funcionamento de hardware e softwares para aperfeiçoar o compartilhamento de informações. O curso tem enfoque especial para artigos e tópicos de pesquisa atuais e de interesse de alunos e do professor.
\subsection{Aplicações desejadas com o conhecimento da disciplina:}
\begin{itemize}
    \item Definir e avaliar arquitetura de rede para uma determinada aplicação;
    \item Instalar e manter redes de computadores;
    \item Gerenciar a implementação de uma rede;
    \item Supervisionar a operação de uma rede;
    \item Avaliar e solucionar problemas em redes de computadores;
\end{itemize}

\section{Relação com outras disciplinas}
A disciplina apresenta apenas uma outra disciplina como pré-requisito e nenhuma outra como co-requisito.
\begin{table}[h]
\begin{tabular}{lllll}
IF678- Infraestrutura de comunicação & \multicolumn{1}{c}{Disciplina pré-requisito} &  &  &  \\
IF574- Redes de computadores 1       & Disciplina equivalente                             &  &  &  \\
IF123- Redes de computadores 2       & Disciplina equivalente                             &  &  &  \\
                                     &                                                    &  &  & 
\end{tabular}

\end{table}

\bibliography{bnm}
\end{document}
